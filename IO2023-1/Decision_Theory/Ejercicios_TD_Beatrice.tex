\documentclass[11pt,letterpaper]{article}
\usepackage[spanish, es-noshorthands]{babel}
\usepackage[left=2.5cm,right=2.5cm,top=2.5cm,bottom=2.5cm]{geometry}
\usepackage[colorlinks, urlcolor=blue]{hyperref}
\usepackage{graphicx}
\usepackage{float}
\usepackage{arydshln}
\usepackage{adjustbox}
\usepackage{listings}
\usepackage{xcolor}
\usepackage{amsmath}
\usepackage{kvmap}
\usepackage{minted}
\usepackage{tikz}
\usepackage{forest}
\usepackage{mathtools}
\tikzstyle{decision} = [rectangle, minimum height=18pt, minimum width=18pt, draw=black, fill=none, thick, inner sep=0pt]
\tikzstyle{chance} = [circle, minimum width=18pt, draw=black, fill=none, thick, inner sep=0pt]
\tikzstyle{line} = [draw=none]

\tikzset{
grow=right,
sloped,
join=miter,
level 1/.style={sibling distance=5cm,level distance=3cm},
level 2/.style={sibling distance=4cm, level distance=3cm},
level 3/.style={sibling distance=2.5cm, level distance=3cm},
level 4/.style={sibling distance=1cm, level distance=3cm},
edge from parent/.style={thick, draw=black},
edge from parent path={(\tikzparentnode.east) -- (\tikzchildnode.west)},
every node/.style={text ragged, inner sep=1mm}
}
\begin{document}

\renewcommand{\listtablename}{Índice de tablas} 
\renewcommand{\tablename}{Tabla} 

\begin{titlepage}
   \begin{center}
       \includegraphics[width=400pt]{Departamento-de-Informtica_HORIZONTAL.png}
       \vspace*{3cm}
        
       \textbf{\LARGE Resolución guía ejercicios:}

       \vspace{0.5cm}
        \Large \textbf{``Teoría de Decisión"}
            
       \vspace{2.5cm}

       \large Beatrice Valdés, 201941556-5

       \vfill
            
       \vspace{0.8cm}
     
       Departamento de Ingeniería Civil Informática\\
       Universidad Técnica Federico Santa María\\
       Investigación de Operaciones INF292\\
       Agosto de 2023
            
   \end{center}
\end{titlepage}


\section{Caso 1}
\begin{itemize}
    \item Objetivo: Maximizar la felicidad del marido
    \item Cursos de acción: $A=\{a_E, a_B, a_H, L, \neg L\}$
        \begin{itemize}
            \item $a_E$ : Cocinar exquisitez
            \item $a_B$ : Cocinar bistec
            \item $a_H$ : Cocinar hamburguesa
            \item $L$ : Llamar al marido
            \item $\neg L$ : No llamar al marido
        \end{itemize}
    \item Estados de la naturaleza: $S=\{s_A, s_F, s_g, r_1, r_2, r_3\}$
        \begin{itemize}
            \item $s_A$ : El marido está alegre
            \item $s_F$ : El marido está fácil de tratar
            \item $s_G$ : El marido está gruñón
            \item $r_1$ : El marido responde 1
            \item $r_2$ : El marido responde 2
            \item $r_3$ : El marido responde 3
        \end{itemize}
    \item Matriz de utilidad: (enunciado)
    \begin{table}[H]
        \centering
        \begin{tabular}{c|c|c|c}
                            & Exquisitez & Bistec & Hamburguesa \\\hline
            Alegre          & 30         & 80     & 60\\ \hline
            Fácil de tratar & 65         & 50     & 90\\ \hline
            Gruñón          & 100        & 70     & 00\\ \hline
        \end{tabular}
    \end{table}
\end{itemize}
Además de todo esto desde el enunciado tenemos las siguientes probabilidades:
\begin{itemize}
    \item $P(s_F): 0.6$
    \item $P(s_G): 0.15$
    \item $P(s_A): 1 - (P(s_F) + P(s_G)) = 0.25$
    \item $P(r_1|s_G) = 0.3$
    \item $P(r_1|s_F) = 0.5$
    \item $P(r_1|s_A) = 0.3$
    \item $P(r_2|s_G) = 0.6$
    \item $P(r_2|s_F) = 0.2$
    \item $P(r_2|s_A) = 0.2$
\end{itemize}
Vamos a calcular lo siguiente utilizando la probabilidad total, ya que sabemos que el marido respondió 2:
\begin{equation*}
    P(r_2) = P(r_2|s_A)\times P(s_A) + P(r_2|s_F)\times P(s_F) + P(r_2|s_G)\times P(s_G)
\end{equation*}
\begin{equation*}
    P(r_2) = 0,2\times 0,25 + 0,2\times 0,6 + 0,6\times 0,15 = 0,26
\end{equation*}
Con este resultado y utilizando Bayes podemos obtener las probabilidades de que esté de algún estado de animo dado que respondió 2:
\begin{equation*}
        P(s_A|r_2) = \frac{P(r_2|s_A)\times P(s_A)}{P(r_2)} = \frac{0,2\times 0,25}{0,26} = 0,19
\end{equation*}
\begin{equation*}
        P(s_F|r_2) = \frac{P(r_2|s_F)\times P(s_F)}{P(r_2)} = \frac{0,2\times 0,6}{0,26} = 0,46
\end{equation*}
\begin{equation*}
        P(s_G|r_2) = \frac{P(r_2|s_G)\times P(s_G)}{P(r_2)} = \frac{0,6\times 0,15}{0,26} = 0,35
\end{equation*}
Luego, 
\begin{equation*}
    E(a_E) = P(s_A|r_2) \times 30 + P(s_F|r_2) \times 65 + P(s_G|r_2) \times 100 = 70,6
\end{equation*}
\begin{equation*}
    E(a_E) = P(s_A|r_2) \times 80 + P(s_F|r_2) \times 50 + P(s_G|r_2) \times 70 = 62,7
\end{equation*}
\begin{equation*}
    E(a_E) = P(s_A|r_2) \times 60 + P(s_F|r_2) \times 90 + P(s_G|r_2) \times 0 = 52,8
\end{equation*}
Con ello tendríamos este árbol de decisión (se realizó solo rama de interés):
\begin{center}
\begin{tikzpicture}[H]
\small
\node[decision]{}
    child{node[line]{}
      edge from parent
        node[above]{$\neg L$}
    }
    child{node[chance]{}
      child{node[decision]{...}
        edge from parent
            node[above]{$r_3$}
      }
      child{node[decision]{$70,6$}
        child{node[chance]{$52,8$}
            child{node[line]{0}
                edge from parent
                    node[above]{$s_G$}
            }
            child{node[line]{90}
                edge from parent
                    node[above]{$s_F$}
            }
            child{node[line]{60}
                edge from parent
                    node[above]{$s_A$}
            }
            edge from parent
            node[above]{$a_H$}
        }
        child{node[chance]{$62,7$}
            child{node[line]{70}
                edge from parent
                    node[above]{$s_G$}
            }
            child{node[line]{50}
                edge from parent
                    node[above]{$s_F$}
            }
            child{node[line]{80}
                edge from parent
                    node[above]{$s_A$}
            }
            edge from parent
            node[above]{$a_B$}
        }
        child{node[chance]{$70,6$}
            child{node[line]{100}
                edge from parent
                    node[above]{$s_G$}
            }
            child{node[line]{65}
                edge from parent
                    node[above]{$s_F$}
            }
            child{node[line]{30}
                edge from parent
                    node[above]{$s_A$}
            }
            edge from parent
            node[above]{$a_E$}
        }
        edge from parent
            node[above]{$r_2$}
      }
      child{node[decision]{...}
        edge from parent
            node[above]{$r_1$}
      }
      edge from parent
            node[below]{$L$}
        };
\end{tikzpicture}
\end{center}
\textbf{Por lo tanto Ana María debe cocinar el menú exquisitez y así obtendrá la máxima utilidad o felicidad con un valor de $70,6$.}\\\\
Para saber en cuanto aumento la utilidad gracias al llamado telefónico debemos resolver la rama en la que no se realizó el llamado, es decir:

\tikzset{
grow=right,
sloped,
join=miter,
level 1/.style={sibling distance=5cm,level distance=3cm},
level 2/.style={sibling distance=3cm, level distance=3cm},
level 3/.style={sibling distance=1cm, level distance=3cm},
level 4/.style={sibling distance=1cm, level distance=3cm},
edge from parent/.style={thick, draw=black},
edge from parent path={(\tikzparentnode.east) -- (\tikzchildnode.west)},
every node/.style={text ragged, inner sep=1mm}
}

\begin{center}
\begin{tikzpicture}[H]
\small
\node[decision]{}
    child{node[line]{}
      edge from parent
        node[above]{$L$}
    }
    child{node[decision]{$69$}
      child{node[chance]{$69,0$}
        child{node[line]{0}
            edge from parent
                    node[above]{$s_G$}
        }
        child{node[line]{90}
            edge from parent
                    node[above]{$s_F$}
        }
        child{node[line]{60}
            edge from parent
                    node[above]{$s_A$}
        }
        edge from parent
            node[above]{$a_H$}
      }
      child{node[chance]{$60,5$}
        child{node[line]{70}
            edge from parent
                    node[above]{$s_G$}
        }
        child{node[line]{50}
            edge from parent
                    node[above]{$s_F$}
        }
        child{node[line]{80}
            edge from parent
                    node[above]{$s_A$}
        }
        edge from parent
            node[above]{$a_B$}
      }
      child{node[chance]{$61,5$}
        child{node[line]{100}
            edge from parent
                    node[above]{$s_G$}
        }
        child{node[line]{65}
            edge from parent
                    node[above]{$s_F$}
        }
        child{node[line]{30}
            edge from parent
                    node[above]{$s_A$}
        }
        edge from parent
            node[above]{$a_E$}
      }
      edge from parent
            node[below]{$\neg L$}
        };
\end{tikzpicture}
\end{center}
Por lo tanto si no llama a su marido, la felicidad tiene un valor de $69$. \textbf{Para calcular el aumento simplemente obtenemos que es $70,6 - 69 = 1,6$}\\\\
Utilizando el criterio Maximin o pesimista tendremos:
\begin{table}[H]
        \centering
        \begin{tabular}{c|c|c|c}
                            & Exquisitez & Bistec & Hamburguesa \\\hline
            Alegre          & 30         & 80     & 60\\ \hline
            Fácil de tratar & 65         & 50     & 90\\ \hline
            Gruñón          & 100        & 70     & 00\\ \hline
            Mínimo          & 30         & 50     & 00\\ 
        \end{tabular}
\end{table}
Luego de encontrar los mínimos debemos seleccionar la mejor opción dentro de ellas. \textbf{Como nuestro objetivo es maximizar la utilidad seleccionamos la con valor $50$, eso es que Ana María debe cocinar bistec según este criterio.}

\section{Caso 2}
\begin{itemize}
    \item Objetivo: Maximizar ganancias
    \item Cursos de acción: $A=\{a_E, \neg a_E, a_I, \neg a_I\}$
        \begin{itemize}
            \item $a_E$ : Realizar el estudio de mercado
            \item $\neg a_E$ : No realizar el estudio de mercado
            \item $a_I$ : Introducir la CocaCola Limón
            \item $\neg a_I$ : No introducir la CocaCola Limón
        \end{itemize}
    \item Estados de la naturaleza: $S=\{s_{EN}, s_{EL}, s_{FN}, s_{FL}\}$
        \begin{itemize}
            \item $s_{EN}$ : La CocaCola Limón es un éxito nacional
            \item $s_{EL}$ : La CocaCola Limón es un éxito local
            \item $s_{FN}$ : La CocaCola Limón es un fracaso nacional
            \item $s_{FL}$ : La CocaCola Limón es un fracaso local
        \end{itemize}
\end{itemize}

El enunciado nos entrega:
\begin{itemize}
    \item $P(s_{EN}): 0.55$
    \item $P(s_{FN}): 0.45$
    \item $P(s_{EL}): 0.6$
    \item $P(s_{FL}): 0.4$
    \item $P(s_{EN}|s_{EL}) = 0.85$
    \item $P(s_{EN}|s_{FL}) = 0.1$
    \item $P(s_{FN}|s_{EL}) = 1 - P(s_{EN}|s_{EL}) = 0.15$
    \item $P(s_{FN}|s_{FL}) = 1 - P(s_{EN}|s_{EL}) = 0.9$
\end{itemize}

Con ello, el árbol de decisión nos queda como:

\tikzset{
grow=right,
sloped,
join=miter,
level 1/.style={sibling distance=5cm,level distance=3cm},
level 2/.style={sibling distance=2cm, level distance=3cm},
level 3/.style={sibling distance=1cm, level distance=3cm},
level 4/.style={sibling distance=1cm, level distance=3cm},
edge from parent/.style={thick, draw=black},
edge from parent path={(\tikzparentnode.east) -- (\tikzchildnode.west)},
every node/.style={text ragged, inner sep=1mm}
}

\begin{center}
\begin{tikzpicture}[H]
\small
\node[decision]{(4)}
    child{node[decision]{(4)}
        child{node[chance]{(4)}
            child{node[line]{B + 300000}
                edge from parent
                    node[below]{$s_{EN}$}
            }
            child{node[line]{B - 100000}
                edge from parent
                    node[above]{$s_{FN}$}
            }
            edge from parent
                node[above]{$a_I$}
        }
        child{node[line]{B}
            edge from parent
                node[above]{$\neg a_I$}
        }
        edge from parent
            node[above]{$\neg a_E$}
    }
    child{node[chance]{(5)}
      child{node[decision]{(3)}
        child{node[line]{$B - 30000 (3)$}
            edge from parent
                    node[below]{$\neg a_I$}
        }
        child{node[chance]{(2)}
            child{node[line]{$B + 300000 - 30000$}
                edge from parent
                    node[below]{$s_{EN}$}
            }
            child{node[line]{$B - 100000 - 30000$}
                edge from parent
                    node[above]{$s_{FN}$}
            }
            edge from parent
                    node[above]{$a_I$}
        }
        edge from parent
            node[above]{$s_{FL}$}
      }
      child{node[decision]{(1)}
        child{node[line]{$B - 30000$}
            edge from parent
                node[below]{$\neg a_I$}
        }
        child{node[chance]{(1)}
            child{node[line]{$B + 300000 - 30000$}
                edge from parent
                    node[below]{$s_{EN}$}
            }
            child{node[line]{$B - 100000 - 30000$}
                edge from parent
                    node[above]{$s_{FN}$}
            }
            edge from parent
                node[above]{$a_I$}
        }
        edge from parent
            node[above]{$s_{EL}$}
      }
      edge from parent
            node[below]{$a_E$}
    };
\end{tikzpicture}
\end{center}
Donde:
\begin{itemize}
    \item $B = 150000000$ (Saldo inicial)
    \item $(1) = P(s_{EN}|s_{EL})\times (B + 300000 - 30000) + P(s_{FN}|s_{EL})\times (B - 100000 - 30000)$\\ $(1) = 150210000$
    \item $(2) = P(s_{EN}|s_{FL})\times (B + 300000 - 30000) + P(s_{FN}|s_{FL})\times (B - 100000 - 30000)$\\ $(2) = 149910000$
    \item $(3) = 149970000$
    \item $(4) = P(s_{EN})\times (B + 300000) + P(s_{FN})\times (B - 100000)$\\ $(4) = 150120000$
    \item $(5) = P(s_{EL})\times (1) + P(s_{FL})\times (3) = 150090000$
\end{itemize}
Luego, como $(4) > (5)$, nos quedamos con la opción número 2 (no hacer el estudio e introducir la CocaCola Limón en el mercado).
\end{document}